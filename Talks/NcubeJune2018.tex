\documentclass{beamer}

\mode<presentation>
{
  \usetheme{default}      % or try Darmstadt, Madrid, Warsaw, ...
  \usecolortheme{default} % or try albatross, beaver, crane, ...
  \usefonttheme{default}  % or try serif, structurebold, ...
  \setbeamertemplate{navigation symbols}{}
  \setbeamertemplate{caption}[numbered]
}



\usepackage[english]{babel}
\usepackage[utf8x]{inputenc}


\usepackage{amssymb}
\usepackage{amsmath}
\usepackage[makeroom]{cancel}

\usepackage{mathtools}
%\usepackage{pst-node}
%\usepackage{auto-pst-pdf}

%\usepackage{tikz-cd}


\newcommand{\ad}[1]{\text{Ad}^{#1}}
\newcommand{\hatad}[1]{\widehat{\text{Ad}}^{#1}}

\newcommand{\lam}[1]{\text{Lam}^{#1}_\bigcirc}
\newcommand{\lamp}[2]{\text{Lam}^{#1}_\bigcirc\left(#2\right)}
\newcommand{\adam}[1]{\text{Ad}^{#1}_{\bigcirc}}
\newcommand{\adamp}[2]{\text{Ad}^{#1}_\bigcirc\left(#2\right)}
\newcommand{\hatadam}[1]{\widehat{\text{Ad}}^{#1}_{\bigcirc}}
\newcommand{\hatadamp}[2]{\widehat{\text{Ad}}^{#1}_{\bigcirc}\left(#2\right)}


\newcommand{\boxlam}[1]{\text{Lam}^{#1}_\square}
\newcommand{\boxlamp}[2]{\text{Lam}^{#1}_\square\left(#2\right)}
\newcommand{\boxadam}[1]{\text{Ad}^{#1}_{\square}}
\newcommand{\boxadamp}[2]{\text{Ad}^{#1}_{\square}\left(#2\right)}
\newcommand{\hatboxadam}[1]{\widehat{\text{Ad}}^{#1}_{\square}}
\newcommand{\hatboxadamp}[2]{\widehat{\text{Ad}}^{#1}_{\square}\left(#2\right)}


\title[Tannakian symbols]{Motives, multiplicative functions, and automated reasoning}
\author{Andreas Holmstrom (Stockholm University)  \\ Torstein Vik (Fagerlia Upper Secondary School, {\AA}lesund} )
\institute{(with contributions by Ane Espeseth, \\ Magnus Hellebust Haaland, Olav Hellebust Haaland)}
\date{N-cube conference, Stockholm, June 2018}


\begin{document}

\begin{frame} \frametitle{\insertsection}
  \titlepage

\end{frame}

%\section{Crash course on lambda-rings}

\section{Intro}

\begin{frame} \frametitle{\insertsection}

We have many new results in number theory - here we hope to give a flavour of the most important ideas.
\vskip20pt
Outline:
\begin{enumerate}
\item Lambda-rings
\item Tannakian symbols
\item Multiplicative functions
\item Motives
\item Miracles
\end{enumerate}

\end{frame}


\section{Crash course on lambda-rings}

\begin{frame} \frametitle{\insertsection}

\uncover<1->{Let $R$ be a torsion-free (unital) commutative ring. }

\uncover<2->{A lambda-structure on $R$ is an infinite sequence of ring homomorphisms $\psi^1$, $\psi^2$, \ldots \ from $R$ to $R$ satisfying the following axioms:}
\vskip10pt
\begin{enumerate}
\uncover<3->{\item $\psi^1(x) = x$ for all $x \in R$.}
\uncover<4->{\item $\psi^m (\psi^n(x)) = \psi^{mn}(x)$ for all $m, n$ and all $x \in R$. }
\uncover<5->{\item $\psi^p(x) \equiv x^p \pmod {pR}$ for all prime numbers $p$ and all $x \in R$.}
\end{enumerate}
\vskip10pt
\uncover<6->{The operations $\psi^1$, $\psi^2$, \ldots are called \textbf{Adams operations}, and we will denote them by $\ad{1}$, $\ad{2}$, etc. }

\end{frame}



\end{document}
